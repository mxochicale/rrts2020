\documentclass[11pt]{article}
\usepackage{biblatex}
\usepackage{datetime2}
\addbibresource{references.bib}

\title{
%Challenges and the future of open-access PhD theses
%%Wed  8 Jul 22:50:42 BST 2020
open-corTeX: A Framework for Open-Accessible Continuous integration for scientific articles
%Sat 25 Jul 15:21:54 BST 2020
}


\author{Author(s)}
\date{\DTMNow}
%(400words)

\begin{document}
\maketitle
Nearly a decade ago, December 2011, Roger Peng published an spectrum of reproducibility 
which depicts a roadmap that went from publication only to full replication as a scientific outcome
and such spectrum went gradually from the publication with code, to publication with code and data to code, 
to publication with code, data and software 
\cite{peng2011}.
Such spectrum was a cornerstone for 
%the launch of
%an engineering-based PhD degree in November 2014
%that, after a four-year journey 
%of time-wise and financial-wise challenges,  
%led to 
the publication of the first open-accessible 
and 100\% reproducible engineering-based PhD thesis in August 2019 
since the establishment of the 
University of Birmingham in 1900 
\cite{xochicale2019-github}.
Nonetheless, there are still many challenges in the existing system 
of formal scientific communication for open-accessible PhD theses.
Such challenges were recently broken down 
by Heise and Pearce 2020
for aspects of evaluation of scientific work, 
speed in the communication process,
respect for the freedom of science and research,
dissemination and accessibility, digitization,
possibilities of verifiability of scientific knowledge, quality, 
and prevention of misuse and scientific misconduct 
\cite{heise2020}.
Alongside with the state-of-the-art in research software engineering
with the use of continuous integration tools (Luger and Foreman-Mackey 2019) 
or the use of containers (Xu 2020)
that contribute with the future of reproducible research 
%\cite{schaduangrat2020}.
%Sustainable development of science
%\cite{pauliuk2020}.
%\cite{spoelstra2013}

In this talk, we will introduce a proof-of-concept of 
"open-corTeX: A framework for Continuously-integrated Open-source Reproducible TeX" 
as an updated version of the spectrum 
of reproducibility for formal scientific communication.
%in the context of open-access theses.
%and other academic documents such as cv, slides, and reports. 
Using open-corTeX, we will present such framework with 
the example of an open-access thesis and 
how the state-of-the-art open-access of scientific communication 
is adopting continuous integration (CI/CD) tools
as well as the use of containers which are impacting reproducibility.
%Additionally, we will show one example of the use of open-corTeX
%with open-cortex that make use of CI/CD tools, containers and 
%its continues releases. 
To then conclude the talk by emphasising that the aim 
of open-corTeX can led to scientific outcomes that are 
aligned the principles of 
reproducibility, inclusiveness, transparency,
reusability  and open accessibility.

\printbibliography
\end{document}
